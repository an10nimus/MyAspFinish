\chapter*{Список сокращений и условных обозначений} % Заголовок
\addcontentsline{toc}{chapter}{Список сокращений и условных обозначений}  % Добавляем его в оглавление
\noindent
%\begin{longtabu} to \dimexpr \textwidth-5\tabcolsep {r X}
\begin{longtabu} to \textwidth {r X}
% Жирное начертание для математических символов может иметь
% дополнительный смысл, поэтому они приводятся как в тексте
% диссертации
\({\boldsymbol{\hat{\mathrm e}}}\) & единичный вектор \\
\(E_0\) & амплитуда падающего поля\\

\end{longtabu}
\addtocounter{table}{-1}% Нужно откатить на единицу счетчик номеров таблиц, так как предыдующая таблица сделана для удобства представления информации по ГОСТ
